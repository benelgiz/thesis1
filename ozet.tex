Do\u{g}ay{\i} tahmin ve elden geldi\u{g}ince kontrol etme istegi insano\u{g}lunun her zaman g\"{u}ndeminde olmu\c{s}tur. Ilk zamanlar \"{o}l\"{u}m kal{\i}m meselesinin bir gere\u{g}i olarak ortaya \c{c}{\i}kan bu g\"{u}d\"{u} zaman i\c{c}erisinde i\c{c}selle\c{s}mi\c{s} ve bizimle b\"{u}t\"{u}nle\c{s}mi\c{s}tir. Art{\i}k, \"{o}l\"{u}m kal{\i}m meselesi olmayan durumlarda dahi do\u{g}ay{\i} kendi faydam{\i}z do\u{g}rultusunda manip\"{u}le etmeye e\u{g}ilimli hale gelmi\c{s} durumday{\i}z. I\c{s}te benzer bir g\"{u}d\"{u}yle, bu tez de nano-par\c{c}ac{\i}klar{\i} geleneksel ak{\i}\c{s}kanlar{\i}n i\c{c}erisine dahil ederek ak{\i}\c{s}{\i}n belli \"{o}zelliklerini istegimiz dogrultusunda de\u{g}i\c{s}tirme ve ak{\i}\c{s}{\i} y\"{o}nlendirmek \"{u}zere manip\"{u}le etmek amac{\i}yla \"{o}zelliklerini ve davran{\i}\c{s}lar{\i}n{\i} anlamak iste\u{g}inden yola \c{c}{\i}kar.

Ferroak{\i}\c{s}kanlar, manyetik \"{o}zelli\u{g}i olmayan ak{\i}\c{s}kanlar{\i}n i\c{c}erisine nano boyutta par\c{c}ac{\i}klar{\i}n kar{\i}\c{s}t{\i}r{\i}lmas{\i}yla elde edilir. Bu yeni ak{\i}\c{s}kan{\i}n {\i}s{\i}l iletim gibi baz{\i} \"{o}zellikleri baz ak{\i}\c{s}kan{\i}n \"{o}zelliklerine nazaran \c{c}ok farkl{\i} ve avantajl{\i} olabilir.  \"{o}zellikle {\i}s{\i}l iletkenlik \"{o}zelli\u{g}inde sa\u{g}lad{\i}klar{\i} belirgin \"{u}st\"{u}nl\"{u}k onlar{\i} pop\"{u}ler bir ara\c{s}t{\i}rma konusu haline getirmi\c{s}tir. Bunlar{\i}n yan{\i} s{\i}ra, d{\i}\c{s}ar{\i}dan uygulanan manyetik alan ile ak{\i}\c{s}kan{\i}n h{\i}z ve s{\i}cakl{\i}k gibi parametrelerini etkin bir \c{s}ekilde de\u{g}i\c{s}tirme olana\u{g}i sa\u{g}larlar. Basta her ne kadar \"{o}l\"{u}m kal{\i}m meselesi gibi g\"{o}r\"{u}nmese de, nanoak{\i}\c{s}kanlar kullan{\i}larak yap{\i}lan t{\i}p alan{\i}ndaki son \c{c}al{\i}\c{s}malar, bu giri\c{s}imlerin \"{o}l\"{u}m kal{\i}m meselesi \c{c}\"{o}zecek a\c{s}amaya getirme yolundad{\i}r. Umut vadeden bir t{\i}p uygulamas{\i} \"{o}rne\u{g}i, tedavi edici manyetik par\c{c}ac{\i}klarla haz{\i}rlanm{\i}\c{s} nanoak{\i}\c{s}kanlar{\i}n, d{\i}\c{s}ar{\i}dan uygulanan bir manyetik alan yard{\i}m{\i}yla kanserli h\"{u}crelere y\"{o}nlendirilmesinin sa\u{g}lanmas{\i}yla sorunsuz h\"{u}crelerin minimum zarar almas{\i}n{\i} sa\u{g}lamak olarak g\"{o}sterilebilir. Is{\i} iletimindeki iyile\c{s}tirme becerileri, her ne kadar alt{\i}nda yatan sebepleri tam olarak henuz \c{c}\"{o}zulememi\c{s} de olsa, farkl{\i} so\u{g}utma uygulamalar{\i}nda kullan{\i}labilir. Bu \c{c}al{\i}\c{s}ma kanal i\c{c}ine enjekte edilmi\c{s} nanopar\c{c}ac{\i}lar{\i}n ak{\i}\c{s}a etkisi \"{u}zerini n\"{u}merik hesaplamalar ile ara\c{s}t{\i}rmaktad{\i}r. Bu sebeple, denklem sistemlerinin verimli bir \c{s}ekilde \c{c}\"{o}z\"{u}lmesi bu \c{c}al{\i}sman{\i}n \"{o}nemli bir par\c{s}as{\i}n{\i} olu\c{s}turmaktad{\i}r. Nanoak{\i}\c{s}lar{\i}n kendine has \"{o}zelliklerinin simulasyonunda deneylere dayal{\i} empirik modeller literat\"{u}rden ara\c{s}t{\i}r{\i}larak kullan{\i}lm{\i}\c{s}t{\i}r.

Bu tez, g\"{o}rece yeni bir ara\c{s}t{\i}rma alan{\i} olan nanoak{\i}\c{s}kan konseptine bir giri\c{s} ve literat\"{u}r taramas{\i}n{\i}n \"{o}zetlenmesi ile ba\c{s}lamaktad{\i}r. Daha sonra, ak{\i}\c{s}{\i} tasvir eden diferansiyel denklemler tan{\i}t{\i}lm{\i}\c{s} ve boyutsuzla\c{s}t{\i}r{\i}lm{\i}\c{s}t{\i}r. Denklemlerin c{c}\"{o}z\"{u}lmesinde kullan{\i}lacak Differential Quadrature Methodunun tan{\i}t{\i}lmas{\i} ile devam eden \c{c}al{\i}\c{s}mada, sim\"{u}lasyonlarla methodun y\"{u}ksek derecede verimli oldu\u{g}u g\"{o}sterilmi\c{s}tir. \"{o}zellikle daha az grid kullan{\i}lmas{\i} gereken durumlarda, method, di\u{g}er n\"{u}merik \c{c}\"{o}z\"{u}m y\"{o}ntemlerine g\"{o}re \"{u}st\"{u}nl\"{u}\u{g}\"{u}n\"{u} kan{\i}tlamaktad{\i}r. Bunun g\"{o}sterilebilmesi i\c{c}in, analitik \c{c}\"{o}z\"{u}m\"{u} literat\"{u}rde var olan ge\c{c}irgen duvarl{\i} kanal problemi nanoak{\i}\c{s}kanlar olmaks{\i}z{\i}n bu y\"{o}ntem ile \c{c}\"{o}z\"{u}lm\"{u}\c{s} ve ba\c{s}ka bir method ile kiyaslanm{\i}\c{s}t{\i}r. De\u{g}i\c{s}ken viskoziteli s{\i}v{\i} i\c{s}eren ve d{\i}\c{s}ar{\i}dan etkiyen bir manyetik alan etkisine maruz kalan bu kanal problemini tasvir eden lineer olmayan diferansiyel denklemler GDQM (Generalized Differential Quadrature method) kullan{\i}larak ayr{\i}kla\c{s}t{\i}r{\i}lm{\i}\c{s}, ve Newton-Raphson methoduyla elde edilmi\c{s} olan denklem setleri \c{c}\"{o}z\"{u}lm\"{u}\c{s}t\"{u}r. Tezde g\"{o}sterilmi\c{s}tir ki, GDQM methodu daha az grid say{\i}s{\i}yla bile di\u{g}er methodlara k{\i}yasla \"{u}st\"{u}n cevaplar vermektedir. Daha sonras{\i}nda, entropi \"{u}retim mekanizmalar{\i} olan manyetik alan, ak{\i}\c{s} s\"{u}rt\"{u}nmesi ve {\i}s{\i} iletimi tart{\i}\c{s}{\i}lm{\i}\c{s} ve Ha say{\i}s{\i} y\"{u}ksek say{\i}lar{\i}nda equapartitioning fenomeni g\"{o}zlenmistir. \c{C}al{\i}\c{s}man{\i}n bu b\"{o}l\"{u}m\"{u}, GDQM'nin, m\"{u}hendislik ve end\"{u}stride bir \c{c}ok uygulamas{\i} olan manyetik alan etkisi alt{\i}nda kanal probleminin c\"{o}z\"{u}m\"{u}nde, i\c{s}lem y\"{u}k\"{u} ac{\i}s{\i}ndan verimli sonuclar verdi\u{g}ini vurgulamay{\i} ama\c{c}lamaktad{\i}r.

\c{C}\"{o}z\"{u}m y\"{o}nteminin secilmesiyle birlikte, kanal problemi geli\c{s}tirilerek icerisine nanoparcaciklar entegre edilmi\c{s}tir. Buradaki ana fikir, manyetik parcaciklar{\i}n ak{\i}\c{s}a dahil edilmesiyle de\u{g}i\c{s}tirilebilen termo-fiziksel \"{o}zelliklere sahip ak{\i}\c{s}lanlar, ferroak{\i}\c{s}kanlar, elde etmekdir. Sabit bir manyetik alan etkisindeki bir e\u{g}ik kanala enjekte edilen ak{\i}\c{s}{\i}n karakteristikleri, GDQM methodu ile c\"{o}z\"{u}lerek ara\c{s}t{\i}r{\i}lm{\i}\c{s}t{\i}r. Bu \c{c}al{\i}\c{s}mada \c{c}al{\i}\c{s}{\i}kan nanoak{\i}\c{s}lan, Cu nanopar\c{c}ac{\i}klar{\i}n{\i}n par\c{c}ac{\i}k oran{\i} 0.06 olacak \c{s}ekilde suya eklenmesiyle elde edilmektedir. Ak{\i}\c{s}{\i}n modellenmesinde enerji denkleminde viskos disipasyon goz \"{o}n\"{u}ne al{\i}nm{\i}\c{s} ve denklemler boyutlu \c{s}ekilde s{\i}n{\i}r ko\c{s}ullar{\i}yla verilmi\c{s}tir. Daha sonras{\i}nda denklemler boyutsuzla\c{s}t{\i}r{\i}lm{\i}\c{s}, ve ad{\i}mlar a\c{c}{\i}k bir c{s}ekilde okuyucuya sunulmu\c{s}tur.  GDQM ile ayr{\i}kla\c{s}t{\i}r{\i}lan denklemler sonucunda elde edilen nonlineer denklem seti, NR metoduyla \c{c}\"{o}z\"{u}lm\"{u}\c{s}t\"{u}r. Bunlar{\i}n yan{\i}nda, manyetik alan{\i}n b\"{u}y\"{u}kl\"{u}\u{g}\"{u}n\"{u}n, e\u{g}ik kanalin e\u{g}ikli\u{g}inin, ve par\c{c}ac{\i}k oran{\i}n{\i}n nanoak{\i}\c{s} h{\i}z ve s{\i}cakl{\i}k profiline etkisi ara\c{s}t{\i}r{\i}lm{\i}\c{s} ve grafiklerle g\"{o}sterilmi\c{s}tir. Manyetik alanin b\"{u}y\"{u}kl\"{u}\u{g}undeki art{\i}\c{s}{\i}n ak{\i}\c{s} h{\i}z{\i}n{\i} muhim bir \c{s}ekilde yava\c{s}latt{\i}\u{g}{\i} ve nanoak{\i}\c{s}kan i\c{c}erisindeki par\c{c}ac{\i}k oran{\i}n{\i}n artmas{\i}yla bu etkinin artt{\i}\u{g}{\i} g\"{o}zenmi\c{s}tir. Kanal e\u{g}imindeki art{\i}\c{s}, ak{\i}\c{s}{\i}n h{\i}z ve s{\i}cakl{\i}\u{g}{\i}n{\i} artt{\i}rm{\i}\c{s}, ancak artt{\i}rma miktar{\i} manyetik alan{\i}n b\"{u}y\"{u}kl\"{u}\u{g}\"{u}n\"{u}n ve nanopar\c{c}ac{\i}k oran{\i}n{\i}n artmas{\i}yla azalm{\i}\c{s}t{\i}r. Nanoparcac{\i}k oran{\i}n{\i}n artmas{\i} ak{\i}\c{s}{\i}n h{\i}z{\i}n{\i} d\"{u}\c{s}\"{u}rmektedir. Manyetik alan{\i}n b\"{u}y\"{u}kl\"{u}\u{g}\"{u} artt{\i}k\c{c}a, h{\i}z{\i}n par\c{c}ac{\i}k oran{\i}na ba\u{g}l{\i}l{\i}\u{g}{\i} da artmaktad{\i}r. S{\i}cakl{\i}k da\u{g}{\i}l{\i}m{\i} nanoparcac{\i}k oran{\i}na fazla ba\u{g}{\i}ml{\i} de\u{g}ildir ve y\"{u}ksek manyetik alan etkisinde bu ba\u{g}lant{\i}n{\i}n daha da azalt{\i}\u{g}{\i} g\"{o}zlenmi\c{s}tir. 

Tezin son b\"{o}l\"{u}m\"{u}nde, e\u{g}ik kanal problemine, a\c{c}{\i}s{\i} ve b\"{u}y\"{u}kl\"{u}\u{g}\"{u} de\u{g}i\c{s}en bir manyetik alan etkidi\u{g}inde ferroak{\i}\c{s}kan{\i}n davran{\i}\c{s}lar{\i} incelenmektedir. Problemi tasvir eden boyutlu nonlineer diferansiyel denklemler verilmi\c{s}tir. Boyutsuzla\c{s}t{\i}rma detaylar{\i}, bir \"{o}nceki probleme benzerli\u{g}i sebebiyle detaylar{\i}yla a\c{c}{\i}klanmaya gerek g\"{o}r\"{u}lmemi\c{s}, boyutsuz denklemler direkt olarak sunulmu\c{s}tur. Boyutsuz denklemler ve s{\i}n{\i}r ko\c{s}ullar{\i} GDQM ile ayr{\i}kla\c{s}t{\i}r{\i}lm{\i}\c{s} ve elde edilen nonlineer denklem seti NR methodu ile \c{c}\"{o}z\"{u}lm\"{u}\c{s}t\"{u}r. Buradaki \c{c}al{\i}\c{s}man{\i}n literat\"{u}re katk{\i}s{\i}, manyetik alan a\c{c}{\i}s{\i} ve kanal e\u{g}iminin ayr{\i} ayr{\i} de\u{g}i\c{s}tirilerek entropi \"{u}retimine etkilerinin incelenmesidir.  Entropi \"{u}retiminin incenmesi optimal sistemlerin tasarlanmasi icin kullanilan bir y\"{o}ntemdir. Entropi \"{u}retimi minimizasyonu, ferroak{\i}\c{s}kan i\c{c}eren e\u{g}ik kanala uygulanm{\i}\c{s} ve de\u{g}i\c{s}ken manyetik alan etkisinin etkileri incelenmi\c{s}tir. Daha \"{o}nceki problemlerde oldu\u{g}u gibi, manyetik alan{\i}n b\"{u}y\"{u}kl\"{u}\u{g}\"{u}n\"{u}n, bu probleme \"{o}zg\"{u}n olarak manyetik alan ac{\i}s{\i}n{\i}n ve nanoparcac{\i}k oran{\i}n{\i}n h{\i}z ve s{\i}cakl{\i}k profillerine etkisi detayl{\i} olarak incelenmi\c{s}tir. Manyetik alan a\c{c}{\i}s{\i} ve kanal e\u{g}im a\c{c}{\i}s{\i}n{\i}n entropi \"{u}retimi \"{u}zerine etkileri ayr{\i} ayr{\i} ara\c{s}t{\i}r{\i}lm{\i}\c{s}t{\i}r. Incelemeler sonunda g\"{o}r\"{u}len odur ki, manyetik alan{\i}n b\"{u}y\"{u}kl\"{u}\u{g}\"{u}ndeki art{\i}\c{s} ak{\i}\c{s} h{\i}z{\i} yava\c{s}latmakta ve bu etki nanoparcac{\i}klar{\i}n oran{\i}n{\i}n artmas{\i}yla artmaktad{\i}r. H{\i}z ve s{\i}cakl{\i}k kanal boyunca kanal{\i}n e\u{g}ikli\u{g}inin y\"{u}kselmesiyle artmakta ancak bu art{\i}\c{s} daha y\"{u}ksek manyetik alan b\"{u}y\"{u}kl\"{u}klerinde ve nanoparcac{\i}k oranlar{\i}nda azalmaktad{\i}r. Manyetik alan{\i}n b\"{u}y\"{u}kl\"{u}\u{g}\"{u} artt{\i}kca, h{\i}z{\i}n par\c{c}ac{\i}k oran{\i}na ba\u{g}l{\i}l{\i}\u{g}{\i} da artmaktad{\i}r. Par\c{c}ac{\i}k oran{\i}n{\i}ndaki de\u{g}i\c{s}imin s{\i}cakl{\i}k alan{\i}na etkisi yok say{\i}lacak kadar azd{\i}r ve artan manyetik alan b\"{u}y\"{u}klerinde daha da azalmaktad{\i}r. Kanal e\u{g}iminin d\"{u}\c{s}\"{u}k de\u{g}erleri i\c{c}in, manyetik alan b\"{u}y\"{u}kl\"{u}\u{g}\"{u}n\"{u}n artmas{\i}yla entropi \"{u}retimi azalmaktad{\i}r. Kanal e\u{g}iminin y\"{u}ksek ac{\i}lar{\i}nda, manyetik alan b\"{u}y\"{u}kl\"{u}\u{g}\"{u}n\"{u}n artmas{\i} entropi \"{u}retimini \"{o}nce d\"{u}\c{s}\"{u}rmekte ancak belli bir de\u{g}erden sonra artt{\i}rmaktad{\i}r. Minimum entropi \"{u}retimi manyetik alanin kanala dik oldu\u{g}u durumlarda g\"{o}zlenmi\c{s}tir.
