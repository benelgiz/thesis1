%1 line spacing must be set for summaries. For theses in Turkish, the summary in
%Turkish must have 300 words minimum and span 1 to 3 pages, whereas the extended
%summary in English must span 3-5 pages.
%For theses in English, the summary in English must have 300 words minimum and
%span 1-3 pages, whereas the extended summary in Turkish must span 3-5 pages.
%A summary must briefly mention the subject of the thesis, the method(s) used and 
%the conclusions derived.
%References, figures and tables must not be given in Summary.
%Above the Summary, the thesis title in first level title format 
%(i.e., 72 pt before and 18 pt after paragraph spacing, and 1 line 
%spacing) must be placed. Below the title, expression {\bf \"OZET} 
%(for summary in Turkish) and {\bf SUMMARY} (for summary in English)
%must be written horizontally centered.
%It is recommended that the summary in English is placed before
%the summary in Turkish.

To predict and control the nature has always been a human desire. Formerly answering the matters of life and death, probably in time has turned into a survival instinct and sticked to us. Now, not necessarily for urgencies, we have a tendency to rule the nature for our needs. With such a motive, this work discusses the addition of nano-sized particles into conventional fluids to acquire tunable properties and flow. Ferrofluids, a mixture of nonmagnetic fluids with magnetic nano-sized particles, offer advantages such as adjustable thermal conductivity and ability to be controlled via an external magnetic field. Although referred to as not a matter of life and death, ferrofluids have various medical applications which might serve as one soon. An example for promising medical application of nanofluids is cancer treatments, where external magnetic field is utilized to guide special magnetic nanoparticles directly to the malign tumors giving less harm to the surrounding tissues. Their not totally understood superiority for heat conductance also offers effective solutions for a variety of different cooling applications.

This work focuses on investigation of flow characteristics of this new engineered fluids in channel flows. Study here involves numerical calculations rather then experimentation. Thus, efficient tools to solve the system equations plays a vital role. Nanoparticle specific properties of the flow has been simulated by the available emprical models in the literature. 

Starting with an introduction of the relatively new topic of nanofluids and a literature survey, an overview of the topic has been aimed. Then, to solve the partial differential equations of motion for the flow, a not widely used but powerful method has been suggested for the solution of the problem. This method is called the Differential Quadrature Method (DQM) and in this work, it proved itself to be a highly efficient and powerful solver when compared to conventional techniques especially for very few grids. To be able to show its effectiveness, a permeable channel problem has been solved via DQM, with no nanoparticles involved for the convenience of literature for comparison. To demonstrate DQM's superiority, a variable viscosity channel flow under constant magnetic field is investigated. Nonlinear, coupled differential equations, representing steady viscous incompressible flow of an electrically conducting fluid between two porous plates under magnetic field, are discretized utilizing Generalized Differential Quadrature Method (GDQM), which is a semi numerical-analytical solution technique and gives serial form solutions, and solved via Newton- Raphson (NR) method. It is shown that, this combination leads to more accurate results compared to available ones in the literature. Furthermore, entropy generation mechanisms due to magnetic field, fluid friction and heat transfer are discussed in detail. An equipartition phenomenon, which corresponds to the degree of freedom of the system, is investigated and it is found that when Ha number takes greater values, equipartition phenomenon is observed. The study highlights that GDQM is a very strong tool to converge to accurate solutions with a few grid points, enabling more computationally efficient solutions for channel flow problems under magnetic field, which has numerous applications in engineering and industry. 

Further on, DQM is utilized for nanofluid flow solutions. Dispersion of superparamagnetic nanoparticles in a nonmagnetic carrier fluid, ferrofluids, offer the advantages of tunable thermophysical properties and eliminate the need for moving parts to induce flow. Following the selection of the efficient numerical tool for solutions, nanofluid flow characteristics of an inclined channel flow exposed to constant magnetic field and pressure gradient is investigated. This part of the study is based on the work done in the author's paper \cite{baskaya2014investigation}. The nanofluid considered is water based Cu nanoparticles with a volume fraction of 0.06. The viscous dissipation is taken into account in the energy equation and the governing differential equations are nondimensionalized. The coupled one dimensional differential equations are solved via Generalized Differential Quadrature Method (GDQM) discretization followed by Newton-Raphson method. Furthermore, the effect of magnetic field, inclination angle of the channel and volume fraction on nanoparticles in the nanofluid on velocity and temperature profiles are examined and represented by figures to give a thorough understanding of the system behavior.  It is observed that an increase in magnetic field density suppresses the flow field significantly, and this effects gets stronger as the volume fraction of nanoparticles in nanofluids increases. The velocity and temperature increases as the inclination of the channel rises but this dependency diminishes with larger magnetic field intensity or volume fraction of nanofluids. The velocity decreases with an increase in volume fraction of nanofluid. When magnetic field gets stronger, the volume fraction dependence of the velocity also increases and gets more dependent to temperature distribution. The temperature distribution is not very effected by volume fraction and this dependence gets even less with increased magnetic field magnitude.

Last part of the thesis investigates ferrofluid flow characteristics in an inclined channel under inclined magnetic field and constant pressure gradient. The ferrofluid considered in this work is comprised of Cu particles as the nanoparticles and water as the base fluid. The governing differential equations including viscous dissipation are non-dimensionalised and discretized with Generalized Differential Quadrature Method. The resulting algebraic set of equations are solved via Newton-Raphson Method. This part of the study is based on the work done in the author's paper \cite{baskaya2017investigation}.The work done here contributes to the literature by searching the effects of magnetic field angle and channel inclination separately on the entropy generation of the ferrofluid filled inclined channel system in order to achieve best design parameter values so called entropy generation minimization is implemented. Furthermore, the effect of magnetic field, inclination angle of the channel and volume fraction of nanoparticles on velocity and temperature profiles are examined and represented by figures to give a thorough understanding of the system behavior.

The effects of magnetic field orientation angle and channel inclination angle are separately investigated, when the channel filled with ferrofluid, on the entropy generation.  And~for such a channel system, related parameters investigated to produce minimum entropy generation case. The viscous dissipations and buoyancy effects are included in the governing equations. Derived governing equations are non-dimensionalized by using physically appropriate parameters. 

Equations for flow and thermal fields are discretized using GDQM, a new computationally efficient tool giving fairly accurate results for even very few grid points. The discretized system of equations are solved simultaneously utilizing Runge--Kutta scheme.
 An increase in magnetic field density suppresses the flow field significantly, and these effects get stronger as the volume fraction of nanoparticles in nanofluids increases. The velocity and temperature increases as the inclination of the channel rises but this tendency diminishes with larger magnetic field intensity or volume fraction of nanofluids. The velocity decreases with an increase in volume fraction of nanoparticles. When~magnetic field gets stronger, the volume fraction dependence of the velocity also increases and gets more dependent to temperature distribution. Influence of a change in particle volume fraction on temperature distribution is minor and diminished with increased magnetic field magnitude. The~entropy generation decreases with increasing magnetic field angle for smaller values of channel inclination. For higher values of channel inclination, with an increase in magnetic field angle, 
the~entropy generation first decreases and then increases. The minimum entropy generation is observed around when the magnetic field angle is perpendicular to the channel. Thus, to further optimize the system by managing velocity and temperature distributions, an multi-objective optimization method should be utilized to serve the users needs.

