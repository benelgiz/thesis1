%%%%%%%%%%%%%%%%%%%%%%%%%%%%%%%%%%%%%%%%%%%%%%%%%%%%%%%%%%%%%%%%%
\chapter{CONCLUSION}\label{ch:ifnecch5}
%%%%%%%%%%%%%%%%%%%%%%%%%%%%%%%%%%%%%%%%%%%%%%%%%%%%%%%%%%%%%%%%%

In this thesis, nanofluid filled channel flows have been studied. Nanofluids offer a variety of promising properties although the underlying principles enabling these advantages not thoroughly understood. Emprical models in the literature are used to model the nanofluid properties such as thermal conductivity. The properties of the flow mainly discussed in terms velocity and temperature profiles due to their effects on thermal management. And then the entropy generation is discussed for designing more efficient systems via minimizing it. So the effects of a variety of parameters on entropy generation is examined. 

This work discusses three different problems, where the two are quite similar. The first problem investigates the GDQM methods performance in solving channel problems. For that purpose, the steady viscous incompressible flow of an electrically conducting fluid between two porous plates under magnetic field is studied. A combination of GDQM \& NR methods is applied to obtain velocity and temperature fields in the channel having constant injection and suction walls. The velocity, temperature fields, local entropy generation rate and total entropy generation are investigated depending on physical parameters such as Ha number, Suction/Injection parameter, V*, function of viscosity variation, $\varepsilon $.

Then, the performance of GDQM method is examined for equal gridding, CGL gridding techniques and different number of grids. The results compared with existing work using Runge-Kutta (RK) method. GDQM is a strong semi numerical solution tool for discretization of partial derivatives. Combining GDQM and Newton-Raphson Method gives more accurate results than solutions in the open literature via RK Method. It is also noticed that with a few grid points in GDQM solution technique, the results obtained are accurate enough. The inherent success of GDQM to converge to accurate solutions using a few number of grid points is also observed in the simulations and pointed out in tables. 

In second part of the work, nanofluids are included in the study. The effect of magnetic field, inclination angle of the channel, and volume fraction on the velocity profile and the temperature distribution are investigated in an inclined channel problem. The governing equations are given including viscous dissipation and bouyancy terms. Utilizing appropriate dimensionless parameters and defining some others, the equations are nondimensionalized.  Equations for flow and thermal fields are discretized using GDQM, giving fairly accurate results for even very few grid points. The discritized system of equations are solved simultanously utilizing Runge- Kutta scheme. 

It is observed that an increase in magnetic field density suppresses the flow field significantly, and this effects gets stronger as the volume fraction of nanoparticles in nanofluids increases. The velocity and temperature increases as the inclination of the channel rises but this dependency diminishes with larger magnetic field intensity or volume fraction of nanofluids. The velocity decreases with an increase in volume fraction of nanofluid. When magnetic field gets stronger, the volume fraction dependence of the velocity also increases and gets more dependent to temperature distribution.  The temperature distribution is not very effected by volume fraction and this dependence gets even less with increased magnetic field magnitude.

Investigation of entropy generation and equipartition constitutes another main argument of this part of the study. Quantification of the destruction of system's available work is a key issue in the design of efficient system. The contribution of each component of entropy generation mechanisms, magnetic field irreversibility, heat transfer irreversibility, fluid friction irreversibility is discussed. Their variation with flow parameters such as Ha number and V* number are investigated. An equipartion phenomenon, equal distribution of entropy generation contributions, is observed between magnetic field irreversibility, heat transfer irreversibility and fluid friction irreversibility for higher values of Ha numbers. The variation of total entropy generation with flow parameters such as Ha number and V* number, is not linear, thus minimizing entropy generation should be handled via an optimization algorithm.

In the last part of the work, the effects of magnetic field orientation angle and channel inclination angle are separately investigated, when the channel is filled with ferrofluid.  And~for such a channel system, related parameters investigated to produce minimum entropy generation such as magnetic field magnitude, channel inclination angle and nanofluid volume fraction. The viscous dissipations and buoyancy effects are included in the governing equations. Derived governing equations are non-dimensionalised by using physically appropriate parameters. 

Equations for flow and thermal fields are discretized using GDQM, a new computationally efficient tool giving fairly accurate results for even very few grid points. The discretized system of equations are solved simultaneously utilizing Runge--Kutta scheme.
 An increase in magnetic field density suppresses the flow field significantly, and these effects get stronger as the volume fraction of nanoparticles in nanofluids increases. The velocity and temperature increases as the inclination of the channel rises but this tendency diminishes with larger magnetic field intensity or volume fraction of nanofluids. The velocity decreases with an increase in volume fraction of nanoparticles. When~magnetic field gets stronger, the volume fraction dependence of the velocity also increases and gets more dependent to temperature distribution. Influence of a change in particle volume fraction on temperature distribution is minor and diminished with increased magnetic field magnitude. The~entropy generation decreases with increasing magnetic field angle for smaller values of channel inclination. For higher values of channel inclination, with an increase in magnetic field angle, 
the~entropy generation first decreases and then increases. The minimum entropy generation is observed around when the magnetic field angle is perpendicular to the channel. Thus, to further optimize the system by managing velocity and temperature distributions, an multi-objective optimization method should be utilized to serve the users needs.
